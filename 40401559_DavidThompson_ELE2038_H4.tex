\documentclass[10pt,a4paper]{article}
\usepackage{student}

% Metadata
\date{\today}
\setmodule{ELE2038: Signals and Control}
\setterm{Semester 2, 2025}

%-------------------------------%
% Other details
\title{Assignment H4}
\setmembername{David Thompson}
\setmemberuid{40401559}

%-------------------------------%
% Commands and packages
\usepackage{amsmath,amssymb,bm,physics,wrapfig}
\usepackage[backend=biber, style=ieee]{biblatex}
\addbibresource{ELE2038_H4.bib}
\nocite{*}

\newcommand{\KL}{\mathrm{KL}}
\newcommand{\R}{\mathbb{R}}
\newcommand{\E}{\mathbb{E}}
\newcommand{\T}{\top}

\newcommand{\expdist}[2]{%
        \normalfont{\textsc{Exp}}(#1, #2)%
    }
\newcommand{\expparam}{\bm \lambda}
\newcommand{\Expparam}{\bm \Lambda}
\newcommand{\natparam}{\bm \eta}
\newcommand{\Natparam}{\bm H}
\newcommand{\sufstat}{\bm u}

%-------------------------------%
% Main document
\begin{document}
    \numberwithin{equation}{section}
    % Add header
    \header{}
    \section{Problem 1} 
        A system with a transfer function
        \begin{equation}
            G(s) = \frac{s^3 + s^2 - 5s -1}{8s^5 + 38s^4 + 65s^3 + 50s^2 + 17s + 2},
        \end{equation}
        can be shown to be BIBO-stable through the use of Routh's criterion. Defining $G$ as $G(s) = \frac{P(s)}{Q(s)}$, we can take $Q$ as the characteristic polynomial and construct a Routh's tabulation using its coefficients as shown in Table \ref{tb:rouths_Q}. When tabulated, it can be seen that all values in the first column are positive, therefore, all roots of $Q$ have a negative real part, meaning $G$ is BIBO-stable. $G$ can also be shown to be "sufficiently stable" if all of its poles are more than $0.2$ from the imaginary axis. We can show this by constructing another Routh's tabulation with the coefficients of $Q(s - c)$ where $c$ is the distance from the axis, in this case, $0.2$. If all values in the first column are positive, the poles of $G(s)$ are more than $0.2$ from the imaginary axis. Using the binomial theorem to expand $Q(s-c)$ and calculate the coefficients, 
        \begin{align}
            Q(s + 0.2) &= 8(s + 0.2)^5 + 38(s + 0.2)^4 + 65(s + 0.2)^3 + 50(s + 0.2)^2 + 17(s + 0.2) + 2 \\
            &= 8s^5 + 30s^4 + 37.8s^3 + 19.48s^2 + 3.648s + 0.138
        \end{align}
        Table \ref{tb:rouths_Qc} shows that this is the case, hence $G$ is BIBO-stable with all poles more than $0.2$ from the axis.
        \begin{table}[h!]
            \centering
            \begin{tabular}{ c | c c c}
                $s^5$ & 8     & 65    & 17 \\
                $s^4$ & 38    & 50    & 2  \\
                $s^3$ & 54.47 & 16.58 & 0  \\
                $s^2$ & 38.43 & 2     & 0  \\
                $s^1$ & 13.74 & 0     & 0  \\
                $s^0$ & 2     & 0     & 0             
            \end{tabular}
            \caption{Routh's tabulation of $Q(s)$.}
            \label{tb:rouths_Q}
            \end{table}
        \begin{table}[h!]
            \centering
            \begin{tabular}{ c | c c c}
                $s^5$ & 8     & 37.80 & 3.648 \\
                $s^4$ & 30    & 19.48 & 0.138 \\
                $s^3$ & 32.61 & 3.611 & 0  \\
                $s^2$ & 16.16 & 0.138 & 0  \\
                $s^1$ & 3.332 & 0     & 0  \\
                $s^0$ & 0.138 & 0     & 0             
            \end{tabular}
            \caption{Routh's tablulation of $Q(s + 0.2)$.}
            \label{tb:rouths_Qc}
            \end{table}

        \section{Problem 2}
        
    \printbibliography
\end{document}